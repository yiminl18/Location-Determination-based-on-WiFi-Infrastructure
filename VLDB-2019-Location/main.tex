% THIS IS AN EXAMPLE DOCUMENT FOR VLDB 2012
% based on ACM SIGPROC-SP.TEX VERSION 2.7
% Modified by  Gerald Weber <gerald@cs.auckland.ac.nz>
% Removed the requirement to include *bbl file in here. (AhmetSacan, Sep2012)
% Fixed the equation on page 3 to prevent line overflow. (AhmetSacan, Sep2012)

\documentclass{vldb}
\usepackage{graphicx}
\usepackage{balance}  % for  \balance command ON LAST PAGE  (only there!)


\vldbTitle{A Sample Proceedings of the VLDB Endowment Paper in LaTeX Format}
\vldbAuthors{Ben Trovato, G. K. M. Tobin, Lars Th{\sf{\o}}rv{$\ddot{\mbox{a}}$}ld, Lawrence P. Leipuner, Sean Fogarty, Charles Palmer, John Smith, Julius P.~Kumquat, and Ahmet Sacan}
\vldbDOI{https://doi.org/10.14778/xxxxxxx.xxxxxxx}
\vldbVolume{12}
\vldbNumber{xxx}
\vldbYear{2019}
% Include information below and uncomment for camera ready
%\vldbTitle{Location Determination based on WiFi Infrastructure}
%\vldbAuthors{Ben Trovato, G. K. M. Tobin, Lars Th{\sf{\o}}rv{$\ddot{\mbox{a}}$}ld, Lawrence P. Leipuner, Sean Fogarty, Charles Palmer, John Smith, Julius P.~Kumquat, and Ahmet Sacan}
%\vldbDOI{https://doi.org/10.14778/xxxxxxx.xxxxxxx}
%\vldbVolume{12}
%\vldbNumber{xxx}
%\vldbYear{2019}

\begin{document}

% ****************** TITLE ****************************************

\title{Location Determination based on WiFi Infrastructure}


% possible, but not really needed or used for PVLDB:
%\subtitle{[Extended Abstract]
%\titlenote{A full version of this paper is available as\textit{Author's Guide to Preparing ACM SIG Proceedings Using \LaTeX$2_\epsilon$\ and BibTeX} at \texttt{www.acm.org/eaddress.htm}}}

% ****************** AUTHORS **************************************

% You need the command \numberofauthors to handle the 'placement
% and alignment' of the authors beneath the title.
%
% For aesthetic reasons, we recommend 'three authors at a time'
% i.e. three 'name/affiliation blocks' be placed beneath the title.
%
% NOTE: You are NOT restricted in how many 'rows' of
% "name/affiliations" may appear. We just ask that you restrict
% the number of 'columns' to three.
%
% Because of the available 'opening page real-estate'
% we ask you to refrain from putting more than six authors
% (two rows with three columns) beneath the article title.
% More than six makes the first-page appear very cluttered indeed.
%
% Use the \alignauthor commands to handle the names
% and affiliations for an 'aesthetic maximum' of six authors.
% Add names, affiliations, addresses for
% the seventh etc. author(s) as the argument for the
% \additionalauthors command.
% These 'additional authors' will be output/set for you
% without further effort on your part as the last section in
% the body of your article BEFORE References or any Appendices.

%\numberofauthors{8} 

%\author{
%% You can go ahead and credit any number of authors here,
%% e.g. one 'row of three' or two rows (consisting of one row of three
%% and a second row of one, two or three).
%%
%% The command \alignauthor (no curly braces needed) should
%% precede each author name, affiliation/snail-mail address and
%% e-mail address. Additionally, tag each line of
%% affiliation/address with \affaddr, and tag the
%% e-mail address with \email.
%%
%% 1st. author
%\alignauthor
%Ben Trovato\titlenote{Dr.~Trovato insisted his name be first.}\\
%       \affaddr{Institute for Clarity in Documentation}\\
%       \affaddr{1932 Wallamaloo Lane}\\
%       \affaddr{Wallamaloo, New Zealand}\\
%       \email{trovato@corporation.com}
%% 2nd. author
%\alignauthor
%G.K.M. Tobin\titlenote{The secretary disavows
%any knowledge of this author's actions.}\\
%       \affaddr{Institute for Clarity in Documentation}\\
%       \affaddr{P.O. Box 1212}\\
%       \affaddr{Dublin, Ohio 43017-6221}\\
%       \email{webmaster@marysville-ohio.com}
%% 3rd. author
%\alignauthor Lars Th{\Large{\sf{\o}}}rv{$\ddot{\mbox{a}}$}ld\titlenote{This author is the
%one who did all the really hard work.}\\
%       \affaddr{The Th{\large{\sf{\o}}}rv{$\ddot{\mbox{a}}$}ld Group}\\
%       \affaddr{1 Th{\large{\sf{\o}}}rv{$\ddot{\mbox{a}}$}ld Circle}\\
%       \affaddr{Hekla, Iceland}\\
%       \email{larst@affiliation.org}
%\and  % use '\and' if you need 'another row' of author names
%% 4th. author
%\alignauthor Lawrence P. Leipuner\\
%       \affaddr{Brookhaven Laboratories}\\
%       \affaddr{Brookhaven National Lab}\\
%       \affaddr{P.O. Box 5000}\\
%       \email{lleipuner@researchlabs.org}
%% 5th. author
%\alignauthor Sean Fogarty\\
%       \affaddr{NASA Ames Research Center}\\
%       \affaddr{Moffett Field}\\
%       \affaddr{California 94035}\\
%       \email{fogartys@amesres.org}
%% 6th. author
%\alignauthor Charles Palmer\\
%       \affaddr{Palmer Research Laboratories}\\
%       \affaddr{8600 Datapoint Drive}\\
%       \affaddr{San Antonio, Texas 78229}\\
%       \email{cpalmer@prl.com}
%}
% There's nothing stopping you putting the seventh, eighth, etc.
% author on the opening page (as the 'third row') but we ask,
% for aesthetic reasons that you place these 'additional authors'
% in the \additional authors block, viz.
%\additionalauthors{Additional authors: John Smith (The Th{\o}rv\"{a}ld Group, {\texttt{jsmith@affiliation.org}}), Julius P.~Kumquat
%(The \raggedright{Kumquat} Consortium, {\small \texttt{jpkumquat@consortium.net}}), and Ahmet Sacan (Drexel University, {\small \texttt{ahmetdevel@gmail.com}})}
%\date{30 July 1999}
% Just remember to make sure that the TOTAL number of authors
% is the number that will appear on the first page PLUS the
% number that will appear in the \additionalauthors section.

%\numberofauthors{8} 
\author{
	Yiming Lin,
	Daokun Jiang,
	Roberto Yus,
	Sharad Mehrotra,
	Nalini Venkatasubramanian \\
	Department of Computer Science\\
	University of California, Irvine\\
	\texttt{\{yiminl18,daokunj,ryuspeir\}@uci.edu}, 
	\texttt{\{sharad,nalini\}@ics.uci.edu}  \\
}
\maketitle

\begin{abstract}

\end{abstract}


\section{Introduction}
\label{sec:intro}

The pervasiveness of WiFi infrastructure enables a large number of services based on it. In this paper we focus on an important location-based service (LBS), which tries to determine the location of device based on WiFi connectivity data. However, such data is ``dirty'' to generate fine-grained location information, i.e., room-level location. Considering a typical scenario, when a device is connected to some WiFi access point (AP), we can conclude that the device is in the region covered by that AP, but no idea of which exact room as a region often covers multiple rooms. 

As an illustrating example, WiFi connectivity data comes in the format of $\{$\textit{Mac Address, WiFi AP, Timestamp}$\}$. As shown in Table~\ref{tab:connectivity}, a device with Mac address \textit{258e} connected to WiFi AP \textit{2} at time ``\textit{2018-03-14 12:03:35}''. Now we can draw that this device must be in one of the rooms covered by $AP_2$, which are  $\{2002,2004,2014,2012,2008,2066,2051,\\2059,2052,2054,2056,2058\}$ (From Fig~\ref{fig:running}) Our goal is to decide the exact room the device is in. 

\begin{figure}[!htb]
	\centering
	\vspace{-0.8em}
	\includegraphics[width=1\linewidth]{running-1}
	\caption{Example of a space with different types of rooms and four WiFi APs.}
	%\vspace{-0.5em}
	\label{fig:running}
\end{figure}

\begin{table}[!htb]
	\centering
%	\vspace{-1em}
	\caption{Sample connectivity data from our experiments.}
	\label{tab:connectivity}
%	\vspace{-1em}
	\label{tab:wifi}
	\begin{tabular}{|c|c|c|}
		\hline
		MAC Address & WiFi AP & Timestamp \\ \hline
		258e... & 2 & 2018-03-14 12:03:35  \\ \hline
		258e... & 3 & 2018-03-14 14:15:53 \\ \hline
		5f0b... & 1 & 2018-03-14 14:16:12 \\ \hline
		5f0b... & 1 & 2018-03-14 14:25:36 \\ \hline
	\end{tabular}
	\vspace{-0.5em}
\end{table} 
\section{Problem Definition}
\section{Room Determination}
\section{Building Determination}
\label{sec:building}
\section{Scalability}
\section{Evaluation}
\section{Related Work}
\section{Conclusion}
\label{sec:conclusion}


	\bibliographystyle{IEEEtran}
	\bibliography{vldb_sample}




\end{document}
