\section{Introduction}
\label{sec:intro}


Locating people within buildings is important for variety of application context, e.g., occupancy based HAVC control, building analytic services
that characterize how buildings are used, and location-based services such as
where an employee is, etc. For many of these applications, locating up to the room level is critical for accuracy. As an instance, locating people in room level enables fine-grained occupancy computation, leading to more precise HAVC control~\cite{afram2014theory}.

 While several indoor localization technologies exist, they either require the active participation of devices (active localization~\cite{deak2012survey}), such as WiFi fingerprinting based systems~\cite{kotaru2015spotfi}, or require extra hardware, such as BlueTooth~\cite{diaz2010bluepass}, RFID~\cite{shirehjini2012rfid}, Visible Light~\cite{kuo2014luxapose}, Ultra Wideband~\cite{gezici2005localization}, acoustic signal~\cite{huang2015swadloon} based, etc. However, they fall short in following aspects, which prevent them from being widely adopted. First, hardware deployment can be prohibitive in terms of cost in large deployments and/or may not be possible due to physical restrictions of space; Second, users are not always willing to cooperate in the localization process as it requires the execution of additional services on their devices which, in turn, might incur additional energy consumption. In addition, users may not bother installing special software either due to inertia or since they may not value the services offered to be important.; Third, the platform and device heterogeneity prevents active localization from being applied to all platforms~\cite{apple}. For instance, Apple does not publish programming APIs for WiFi scanning on iOS-based devices~\cite{zeinalipour2017anatomy} hence, making active WiFi localization hard to achieve for third-party applications on iOS-based devices. 

In contrast WiFi Access Point (AP) is passive, i.e., WiFi connectivity data is generated using SNMP traps over existing WiFi routers and access points. Note that all the passive localization work so far requires extra equipment, such as WiFi monitors~\cite{youssef2007challenges,xu2013scpl,seifeldin2013nuzzer,musa2012tracking} and thus it owns similar drawback as the above analysis. 

In this paper, we study a method that uses WiFi data to get location data up to room level, which do not require any extra hardware or explicit cooperation of the user device. Since WiFi connectivity is ubiquitous, if we can use it to locate up to room level reliably, it can serve as a vehicle for several applications. 

Exploiting WiFi AP connectivity, however, is challenge for several reasons:
a) devices only connect sporadically. A missing data might not actually mean user not there; 
b) A WiFi AP covers several rooms, and thus a connection to an AP only tells you the several candidate rooms but not exact one. 
c) WiFi connectivity data often comes at large volume and high speed (stream), and often involves great number of devices, thus efficiency and scalability have to be taken into account. 
We postulate the problem as that of \textit{WiFi data cleaning} problems. 
%explain that problem
We conclude WiFi connectivity data as dirty for LBS in the following two aspects. First,  there exists some time interval where no connectivity is captured by WiFi APs. We need to identify if the device is out of the building or disconnected but in the building, which is formulated as xxx. \footnote{daokun: to summary dirty data from the aspect of gaps Question: Can we call gap dirty data?}; Second, it is hard to generate fine-grained location information, i.e., room-level location. Considering a typical scenario, when a device is connected to some WiFi access point (AP), we can conclude that the device is in the region covered by that AP, but no idea of which exact room as a region often covers multiple rooms. The goal is find the exact room among candidates, which could be viewed as a \textit{value disambiguation} problem. 
As an illustrating example, WiFi connectivity data comes in the format of $\{$\textit{Mac Address, WiFi AP, Timestamp}$\}$. As shown in Table~\ref{tab:connectivity}, for device with Mac Address \textit{5f0b}, it is captured by $AP_{1}$ at ``14:16:12'' and ``14:25:36 '', but we do not know the location information between them. Another device with Mac address \textit{258e} connected to WiFi AP \textit{2} at time ``\textit{2018-03-14 12:03:35}''. Now we can draw that this device must be in one of the rooms covered by $AP_2$, which are  $\{2002,2004,2014,2012,2008,2066,2051,2059,2052,2054,\\2056,2058\}$ (From Fig~\ref{fig:running}) Our goal is to decide the exact room the device is in. 

\begin{figure}[!htb]
	\centering
	\vspace{-0.8em}
	\includegraphics[width=1\linewidth]{running-1}
	\caption{Example of a space with different types of rooms and four WiFi APs.}
	%\vspace{-0.5em}
	\label{fig:running}
\end{figure}

\begin{table}[!htb]
	\centering
	%	\vspace{-1em}
	\caption{Sample connectivity data from our experiments.}
	\label{tab:connectivity}
	%	\vspace{-1em}
	\label{tab:wifi}
	\begin{tabular}{|c|c|c|}
		\hline
		MAC Address & WiFi AP & Timestamp \\ \hline
		258e... & 2 & 2018-03-14 12:03:35  \\ \hline
		258e... & 3 & 2018-03-14 14:15:53 \\ \hline
		5f0b... & 1 & 2018-03-14 14:16:12 \\ \hline
		5f0b... & 1 & 2018-03-14 14:25:36 \\ \hline
	\end{tabular}
	\vspace{-0.5em}
\end{table} 

%Explain overview of approach and explain how well you do.
To address the above challenges in developing fine grained location based on AP data, our approach exploits contextual information that is present in the connectivity data and/or metadata about users. In particular, we leverage knowledge from historical connectivity data such as the APs to which a user device typically connects, which other devices are connected to the same APs at the same time, and the affinity of a user to specific areas in the space. This knowledge can be extracted from spaces where their inhabitants have predictable patterns (e.g., an office or a university building), which are the main scope of our approach. Leveraging such models, our system is able to calculate the probability of a device being present in the space but disconnected from the network. Furthermore, for such events, our approach can compute a prediction on which AP the disconnected device would be connected to based on the context such as the day and time. Then, it predicts the room in which the device would be located given the AP to which it is connected or predicted to be connected. This is done based on a metric of affinity among devices and between devices and rooms based on which we propose a graph-based algorithm for prediction. 

Sensor data cleaning has been considered in the past, such as RFID data cleaning~\cite{jeffery2006adaptive,baba2016learning}. \footnote{daokun: explain how our problem differs with them in concise words}



%-- lots of work on indoor localization, 
%For example, blue tooth beacons, or ultrawide band localization, wifi triagulation etc. 

%-- however, it requires instrumentation either of the building, or user, or both and difficult to achieve. Even for
%wifi triagulation/fingerprinting user has to participate.

%In contrast wifi ap stuff is passive 
%Essentially user connect to wifi AP, that raises an event which  can be captured using snmp traps -- explain.


%-- A passive strategy is through connectivity data of wifi. wifi routers generate data and through snmp traps one can get coarse level location of person. While there are apps which require fine grained location, many apps have requirements only for room level location. \cite{give references}. Explain why this is enough for some apps.

%In this paper, we consider the problem of using wifi connectibity to get location data. 


%The pervasiveness of WiFi infrastructure enables a large number of services based on it. In this paper we focus on Location Based Service (LBS).
%%LBSs use the location of a device (e.g., a smart-phone)  and  hence  of  the  owner  to  offer  personalized  information. Thousands of interesting applications have been built based on such localization techniques.~\cite{akyildiz2002survey}
%However, WiFi data is far from enough to provide location information with high quality if we use it directly. 



%The cleaning of dirty WiFi data poses several challenges. First, predicting the possible location of ... \footnote{daokun: describe challenges in your part}; Second, value disambiguation for room-level location is very tricky only based on WiFi connectivity data as a-priori knowledge is limited. Third, since WiFi data comes on streams, the technique has to be very efficient to support online location determination; Finally, the number of devices captured by WiFi APs within a builiding could be large, so that we need to address scalability issue very carefully to be applicable in large systems. 



In summary, we make the following contributions. a) We first formally define the WiFi data cleaning problem and split it into three subtasks in building, region and room level; b) we propose a hybrid location determination  algorithm that combines graph-based and rule-based methods to exploit contextual information to disambiguate the room a device is located in; c) \footnote{daokun: summary contributions of your part}; d) We optimize the above techniques and make it scale to large number of devices by proposing on-line location determination; e) We show experiments to validate our approach using WiFi connectivity data captured from the 64 WiFi APs in our six-story building over a period of one year. 



The rest of paper is structured as follows. Section~\ref{sec:def} defines problems formally. Section~\ref{sec:room} and ~\ref{sec:building} describes the techniques to clean WiFi data in building, region and room levels. Scalability issues are discussed in Section~\ref{sec:scalability}. Section~\ref{sec:evaluation} evaluates our methods, Section~\ref{sec:related} presents the related work and Section~\ref{sec:conclusion} concludes paper. 